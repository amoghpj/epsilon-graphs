% Created 2023-11-19 Sun 08:47
% Intended LaTeX compiler: pdflatex
\documentclass[11pt]{article}
\usepackage[utf8]{inputenc}
\usepackage[T1]{fontenc}
\usepackage{graphicx}
\usepackage{grffile}
\usepackage{longtable}
\usepackage{wrapfig}
\usepackage{rotating}
\usepackage[normalem]{ulem}
\usepackage{amsmath}
\usepackage{textcomp}
\usepackage{amssymb}
\usepackage{capt-of}
\usepackage{hyperref}
\usepackage{xcolor}
\definecolor{bg}{rgb}{0.9,0.9,0.9}
\usepackage[bottom=0.5in,margin=1in]{geometry}

%%% Leave comments using the syntax \sushrut{your comment here}
\newcommand{\amogh}[1]{{\textsc{\color{blue} AJ:#1}}}
\newcommand{\todo}{{\color{red}{\small TODO\ }}}
\newcommand{\prog}{{\color{orange}{\small PROG\ }}}
\newcommand{\done}{{\color{green}{\small DONE\ }}}
\newcommand{\sushrut}[1]{{\textsc{\color{green}SK:#1}}}

\author{Amogh Prabhav Jalihal and Sushrut Karmalkar}
\date{\today}
\title{\amogh{wip} The presence of intermediate cell states in scRNAseq data obscure the identification of `Waddington' attractors }
\hypersetup{
 pdfauthor={Amogh Prabhav Jalihal and Sushrut Karmalkar},
 pdftitle={\(\epsilon\)-graphs reveal the presence of intermediate cell states on the Waddington landscape scRNAseq datasets},
 pdfkeywords={},
 pdfsubject={},
 pdfcreator={Emacs 27.1 (Org mode 9.4.4)}, 
 pdflang={English}}
\begin{document}

\maketitle
\section*{Outline}
Sequence of arguments:
\begin{itemize}
\item \todo Intro: Small discussion summarizing the Deeds preprint \cite{sparta22}.
\item \todo Result 1a: Overview cartoon of Deed's expectation and a cartoon of how we think about it.
\item \prog Result 1b: Show that the expectations of plateaus in the $\epsilon$-graphs from clustered data is not generally true, for multiple clusters etc.
\item \todo Result 1c: Do a simulation to show that there is some kind of a transition from having plateaus to having a smooth sigmoidal curve as the space between the clusters starts to get filled up with intermediate cell states. \amogh{For sush} Point out that as along as there exists an $\epsilon$-path in this graph, the GCC trend will not show the expected plateau.
\item \todo Result 2: Use real world datasets to examine the epsilon trend.
\item \todo Result 3:  Do a slightly more thorough search to find \textsc{MINCUT} like ideas for identifying cell clusters
\end{itemize}


 
\section*{Introduction}
\label{sec:orgc8a6b0e}
\section*{Results}
\label{sec:org8da0183}
\section*{Discussion}
\label{sec:org4e03e49}

\section*{Methods}
\label{sec:org7a0bf76}
\bibliographystyle{plain}
\bibliography{references}
\end{document}
%%% Local Variables:
%%% mode: latex
%%% TeX-master: t
%%% End:
